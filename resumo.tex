%=======================================================================
% Resumo em Português.
%
% A recomendação é para 150 a 500 palavras.
%=======================================================================
\begin{abstract}

\noindent O presente trabalho tem como objetivo apresentar o paradigma de programação funcional e compará-lo com o paradigma orientado a objetos, amplamente utilizado atualmente. A comparação abrange desde os primórdios de ambos os paradigmas até o cenário atual. Apesar dos avanços significativos no desenvolvimento de software, torna-se necessário analisar diferentes paradigmas de programação para melhorar a qualidade dos produtos, reduzir efeitos colaterais e custos. Foi realizada uma revisão extensiva da literatura, incluindo artigos acadêmicos, livros, publicações especializadas. Com base nessa pesquisa, foi conduzida uma análise comparativa, abordando aspectos como imutabilidade, uso de funções puras, abstração e encapsulamento. Além disso, foram desenvolvidas duas Web APIs, uma em cada paradigma, para resolver o mesmo problema, demonstrando as implicações práticas de cada abordagem em termos de código, desempenho e facilidade de manutenção. A partir dos resultados, foi realizada uma análise crítica, destacando as vantagens do paradigma funcional em comparação ao orientado a objetos em diferentes cenários, considerando fatores como escalabilidade, complexidade de aprendizado e manutenção a longo prazo.
    
\end{abstract}