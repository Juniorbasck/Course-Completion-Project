%=======================================================================
% Exemplo de Apêndice
% O Apêndice é utilizado para apresentar material complementar elaborado
% pelo próprio autor.  Deve seguir as mesmas regras de formatação do
% corpo principal do documento.
%=======================================================================
% \appendix
% \chapter{Informações Complementares}

% Escreva aqui.

%=======================================================================
% Exemplo de Anexo
% O Anexo é utilizado para a ``inclusão de materiais não elaborados pelo
% próprio autor, como cópias de artigos, manuais, folders, balancetes, etc.
% e não precisam estar em conformidade com o modelo''.
%=======================================================================
% \annex
% \chapter{Artigos Publicados}

\bibliography{exemplo}


\appendix
\chapter{Apêndice B}

Este apêndice apresenta os links para os repositórios GitHub das APIs desenvolvidas neste trabalho, permitindo acesso ao código-fonte e à documentação.

\section*{API em Java}
O repositório da API desenvolvida em Java pode ser acessado no seguinte link:
\begin{itemize}
    \item \url{https://github.com/Juniorbasck/API-for-news-management-java}
\end{itemize}

\section*{API em Clojure}
O repositório da API desenvolvida em Clojure pode ser acessado no seguinte link:
\begin{itemize}
    \item \url{https://github.com/Juniorbasck/API-for-news-management-clojure}
\end{itemize}

Para cada repositório, há um arquivo README que explica o funcionamento das APIs, suas funcionalidades e instruções para execução local. Esses repositórios estão disponíveis publicamente para consulta e testes.
